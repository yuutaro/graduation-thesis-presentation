\section{序論}
\label{chap:introduction}

\subsection{背景}

\subsubsection{クリエイターエコノミーの拡大と作品流通の変容}
近年,インターネット技術の発展とソーシャルネットワーキングサービスの普及により,個人が自身の創作物を発信・販売し,収益を得るクリエイターエコノミーが急速に拡大している\cite{goldman2023creator}.動画投稿,ライブ配信,ハンドメイド作品の販売,デジタルコンテンツの提供など,その形態は多岐にわたり,従来は企業や専門家が独占していた生産・流通・販売のプロセスが個人レベルまで民主化された.

特に,物理的な創作物を制作するクリエイターにとって,Mercari \cite{mercari} やminne \cite{minne} といったCtoCマーケットプレイスの普及,あるいはBASE \cite{base} のように個人が手軽に独自のオンラインストアを開設できるサービスの登場は,販路開拓のハードルを劇的に引き下げた.
中でも,イラストや同人誌などのサブカルチャー領域に特化したBOOTH \cite{booth} は,物理的な作品とデジタルコンテンツを並列して扱うことができ,本研究が対象とするような「作品の文脈」を重視するクリエイター層から厚い支持を得ている.
これにより,趣味の延長線上で制作を行っていた個人が,容易に売り手として市場に参加できる環境が整った.

しかし,これらの既存プラットフォームの多くは,既製品や大量生産品,あるいは不用品の流通に最適化されて設計されている側面が強い.商品自体は代替可能な商品として扱われ,その背後にある制作者の思想,制作過程の試行錯誤,作品が生まれた文脈は,商品説明欄の補助的なテキスト情報として扱われるに留まる.

一点物や極めて少数の生産数しか持たない創作物において,その価値の源泉は物理的な素材や機能性よりも,この物語にこそ存在する.既存のプラットフォームでは,こうした作品固有の物語性や文脈を十分に表現・保存することが難しく,結果として作品が単なる商品として消費され,制作者のブランディングや持続的なファンコミュニティの形成に繋がりにくいという課題がある.

\subsubsection{物理的な展示・即売会における制約とアーカイブの欠如}
デジタルプラットフォームの対極として,学園祭,展示会,即売会といった物理的な場での作品発表も依然として重要な役割を果たしている.これらの場は,制作者と鑑賞者が直接交流し,作品の熱量を共有できる貴重な機会である.

しかし,物理的なイベントには一過性という避けられない制約が存在する.
第一に,展示情報のアーカイブ化の問題である.展示会のために制作された解説パネルやキャプションボードは,イベント終了とともに廃棄または散逸することが多く,そこに記された詳細な作品情報はデジタル空間に蓄積されない.結果として,イベントに参加できなかった潜在的なファンには作品の魅力が届かず,制作者のポートフォリオとしても十分に機能しない場合が多い.

第二に,金銭授受と販売機会の損失である.学園祭や小規模な展示会では,販売機能を持たず展示のみとするケースや,販売する場合でも現金取引のみの対応となるケースが多い.また,展示されている作品そのものを購入したいという需要があっても,会期終了後の引き渡し手続きや配送手配の煩雑さから,その場での成約に至らないことも頻繁に発生する.このように,物理的な展示空間とデジタル上の決済・物流システムが分断されていることにより,多くの経済的機会が損失しているのが現状である.

\subsubsection{一点物取引における価格決定と信頼性の課題}
一点物,あるいは極めて供給量の少ない作品の流通においては,価格決定メカニズムと取引の信頼性に関して特有の課題が存在する.

まず,価格決定においては,需要が供給を大幅に上回る場合,定価販売形式では早い者勝ちとなり,転売目的の購入や,botによる自動購入などの問題を引き起こしやすい.これに対し,オークション形式は需要に応じた適正価格を発見する有効な手段であるが,既存のオークションサイトは,前述の通り不用品処分の文脈が強く,クリエイターが自身の作品を発表・販売する場としてのブランディング機能に欠ける場合がある.

\subsubsection{共同体による創作活動と収益分配の複雑性}
創作活動は必ずしも個人だけで完結するものではない.サークル活動,合同誌,アンソロジー企画,YouTuber同士のコラボレーションなど,複数のクリエイターが関与するプロジェクト単位での活動も一般的である.
しかし,既存のCtoCプラットフォームの多くは一人の売り手対一人の買い手という単純な取引モデルを前提としている.そのため,一つの作品またはイベントから得られた収益を,関与した複数のメンバーに分配する場合,代表者が一度全額を受け取り,その後銀行振込などで個別に再分配するという,極めてアナログで煩雑な事務作業が必要となる.これは,企画主催者の負担を増大させ,継続的な共同制作活動を阻害する要因となっている.



\subsection{開発上の課題:複雑な文脈を持つシステムの検証難易度}
前述したような作品の物語や複雑な権利・収益構造を持つプラットフォームを新規に開発する際,最大の障壁となるのが検証用データの不足である.

一般的なECサイト(商品と価格のみの管理)であれば,ランダムな文字列やダミー画像を用いた機械的なデータ生成(Faker.js \cite{fakerjs} 等)で十分に機能検証が可能である.
しかし,本研究が対象とするような文脈を重視するシステムにおいては,以下の理由により,既存のテストデータ生成手法が通用しない.

\begin{enumerate}
	\item \textbf{リレーションの整合性}: プロジェクト主催者と参加メンバー,そして出品された作品の間には,矛盾のない論理的な関係性が求められる.ランダム生成では,「部外者が他人のプロジェクトの作品を管理している」といった矛盾が頻発し,権限管理ロジックのテストが成立しない.
	\item \textbf{検索・推薦アルゴリズムの検証}: 文脈に基づく検索システムを評価するには,データ自体に意味のあるテキスト(制作背景や動機)が含まれている必要がある.Lorem Ipsumのような無意味なテキストでは,検索精度の検証が不可能である.
	% \item \textbf{コールドスタート問題}: 近年のAIを用いたデータ生成手法は,既存の大量のデータを学習元として必要とする.しかし,新規サービスの開発初期段階(0 $\to$ 1フェーズ)においては,学習すべき正解データそのものが存在しない.
\end{enumerate}

したがって,開発者は「手動で時間をかけて整合性のあるデータを作る」か,「質の低いデータで妥協して検証を行う」かの二者択一を迫られており,これが開発効率と品質向上のボトルネックとなっていた.

\subsection{本研究の目的}
本研究の目的は,一点物売買プラットフォームの開発と,それを支える新たなテストデータ生成手法の確立の2点にある.

\subsubsection{一点物作品のナラティブを保存するプラットフォームの開発}
第一に,既存のEコマースでは埋没しがちな作品の文脈を構造化データとして保存・活用できる,新たなCtoCプラットフォームを構築する(第4章にて詳述).これは本研究において,提案するデータ生成手法の有効性を検証するための実験環境としての役割も担う.

\subsubsection{LLMを用いた社会的シナリオ生成によるシステム検証}
第二の,そして本論文の主眼となる目的は,上記のような複雑なリレーションを持つWebサービスの開発において,大規模言語モデル(LLM)を活用した社会的シナリオに基づくテストデータ生成手法を提案・実証することである.

本研究では,LLMに「架空のユーザーの人格」や「プロジェクトの動機」といった社会的背景(社会的シナリオ)を与え,自律的にデータベースへの登録を行わせるエージェントシステムを構築する.
さらに,エージェントに記憶機構(RAG)を組み込むことで,既存データとの衝突を回避し,多様なシナリオを生成できるかを実験的に検証する.
これにより,学習データが存在しない環境下においても,あたかも人間が実際に活動しているかのような,高密度で整合性の取れたテストデータを大量生成できることを示す.
