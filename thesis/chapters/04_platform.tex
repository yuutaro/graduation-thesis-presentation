\section{実験環境:一点物売買プラットフォーム}
\label{chap:platform}

本研究では,提案するデータ生成エージェントの有効性を検証するための実験環境として,一点物の取引に特化したWebプラットフォームを実際に開発した.本章では,このプラットフォームの概要,技術構成,およびテストデータに求められる要件について述べる.

\subsection{プラットフォームの概要}
開発したプラットフォームは,クリエイターが制作した「作品」を制作背景とともに販売・購入できるCtoCマーケットプレイスである.
AmazonやMercari等の代表的なサービスが,商品をスペックと価格で比較される代替可能なモノとして扱うのに対し,本プラットフォームは,作品を代替不可能なアート作品として扱うことを主眼としている.

主な特徴は以下の通りである.
\begin{itemize}
	\item \textbf{プロジェクトベースの活動}: ユーザーは単独で出品するだけでなく,プロジェクトを立ち上げ,複数人で協力して制作・販売を行うことができる.
	\item \textbf{制作ログの重視}: 作品の完成形だけでなく,使用した素材,制作中の失敗,試行錯誤の履歴をログとして記録し,作品価値の一部として提示する.
	\item \textbf{複雑な権利・収益分配}: 一つの作品に対し,素材提供者,制作者,プロジェクト主催者など複数のステークホルダーが存在し,売上はStripe Connectを通じて自動的に分配される.
\end{itemize}

\subsection{システム構成}
本プラットフォームは,拡張性と保守性を重視し,モダンなWeb技術スタックを用いて構築された(図\ref{fig:platform_architecture}).

\begin{figure}[t]
	\centering
	\resizebox{0.95\textwidth}{!}{%
		\begin{tikzpicture}[
				node distance=1.5cm and 3.5cm,
				box/.style={draw, rectangle, rounded corners, align=center, minimum width=3.5cm, minimum height=2cm, fill=white, thick},
				db/.style={draw, cylinder, shape border rotate=90, aspect=0.25, align=center, minimum width=2.5cm, minimum height=2cm, fill=white, thick},
				cloud/.style={draw, ellipse, align=center, minimum width=3cm, minimum height=1.5cm, fill=white, thick},
				arrow/.style={-Latex, thick, <->},
				label_text/.style={font=\footnotesize, align=center}
			]
			% Nodes
			\node[box] (frontend) {\textbf{Frontend / BFF}\\(Next.js)};
			\node[box, right=of frontend] (backend) {\textbf{Backend API}\\(NestJS)};
			\node[db, below=of backend] (database) {\textbf{Database}\\(PostgreSQL)};
			\node[cloud, right=of backend] (stripe) {\textbf{Payment Infra}\\(Stripe Connect)};

			% Connections
			\draw[arrow] (frontend) -- node[midway, above, label_text] {REST / Server Actions} (backend);
			\draw[arrow] (backend) -- node[midway, left, label_text] {Prisma ORM} (database);
			\draw[arrow] (backend) -- node[midway, above, label_text] {SDK / Webhook} (stripe);

			% User Interaction
			\node[left=1.5cm of frontend] (user) {\textbf{User}};
			\draw[arrow, ->] (user) -- node[midway, above, label_text] {HTTPS} (frontend);

		\end{tikzpicture}
	}
	\caption{Payment Platform 2 のシステム構成}
	\label{fig:platform_architecture}
\end{figure}

\subsubsection{フロントエンド}
ユーザーとの接点となるフロントエンドには,Reactベースのフレームワークである Next.js (App Router) \cite{nextjs} を採用した.本システムでは,単なる画面描画だけでなく,サーバーサイドでのデータフェッチや認証セッション管理を行うBFFとしての役割も担わせている.

\paragraph{ディレクトリベース・ルーティングの採用}
ソースコード構成(src/app配下)において,ファイルシステムの階層構造をそのままURLパスにマッピングするディレクトリベース・ルーティングを採用した.
例えば,クリエイターのポートフォリオページは src/app/[userPubId]/page.tsx,作品詳細ページは src/app/[userPubId]/[itemPubId]/page.tsx として配置されている.このように動的パラメータをディレクトリ名に含めることで,https://domain.com/user123/item456 のような,直感的で可読性の高いURL構造を容易に実現している.

\paragraph{Server Actionsとデータ取得の最適化}
コンポーネント設計においては,React Server Components (RSC) を全面的に採用した.作品情報(Markdownテキスト)やユーザープロファイルなどの静的な初期データは,サーバーサイドでデータベース(またはバックエンドAPI)から直接取得し,HTMLとしてクライアントに送信される.
また,フォーム送信や入札アクションなどの動的な操作には Server Actions (src/actions/) を用いることで,クライアントサイドに露出するAPIエンドポイントを隠蔽しつつ,型安全な関数呼び出しとしてバックエンドとの通信を実装している.

\paragraph{ 認証と認可}
認証基盤には Node.jsのライブラリであるNextAuth.js (src/api/auth/[...nextauth]) を導入し,Google OAuthおよびメールアドレス認証によるセキュアなログインフローを構築した.セッション情報はJWT(JSON Web Token)として暗号化され,HTTP Only Cookieを用いて管理される.

\subsubsection{バックエンド}
システムのコアロジックを担当するバックエンドには,Node.jsフレームワークである NestJS \cite{nestjs} を採用した.このフレームワークは、複雑なCtoC取引のステート管理や,金銭に関わる厳密なバリデーションを行うため,TypeScriptによる静的型付けと,依存性の注入(DI)によるモジュラーアーキテクチャを徹底している.

\paragraph{モジュール分割による関心の分離}
アプリケーションは機能単位でモジュールに分割されている.
\begin{description}
	\item[User Module] アカウント管理,プロフィール更新
	\item[Auth Module] 認証処理
	\item[Item Module] 作品のCRUD,Markdown解析,在庫管理
	\item[Order Module] 注文作成,決済ステータスの遷移管理
	\item[Project Module] 企画のガバナンス管理,収益分配計算
\end{description}
このようにドメインごとに責務を分離することで,コードの可読性を高めるとともに,特定の機能(例:オークション処理)に変更が生じた際の影響範囲を最小限に抑えている.

\paragraph{非同期イベント処理}
本システムでは決済処理をStripeに委譲しているため,決済の成功・失敗や,オークションにおける与信枠の確保の結果は,すべて非同期のWebhookイベントとして通知される.
これを受け取るための専用のエンドポイント (src/webhook) を実装し,Stripeからの署名検証を行った上で,Order データベースのステータスを安全に更新する仕組みを構築した.これにより,ネットワーク遅延や一時的な障害が発生しても,最終的なデータの整合性が保証される.

\subsubsection{データベース}
データの永続化には,リレーショナルデータベースである PostgreSQL \cite{postgresql} を採用した.複雑な多対多のリレーション(例:ProjectとCollaborators,ItemとBids)を矛盾なく管理し,ACID特性(原子性,一貫性,独立性,永続性)を備えたトランザクション処理を実現するためである.

\paragraph{Prisma ORMによる型安全なデータアクセス}
アプリケーション層とデータベース層の架け橋として Prisma \cite{prisma} を使用している.schema.prisma ファイルに定義されたデータモデルに基づき,型定義ファイル(TypeScript型)が自動生成される.
これにより,バックエンドの開発において「存在しないカラムへのアクセス」や「型の不一致」といった単純なミスをコンパイル時に検出可能となり,開発効率と品質が大幅に向上した.

\paragraph{スキーマ駆動開発}
開発フローにおいては,まず schema.prisma でデータ構造(User, Item, Project等のモデル定義)を記述し,それを元にマイグレーション(prisma migrate)を実行してDBスキーマを更新する手法をとった.これにより,仕様書に定義されたデータモデルが,実装コードとデータベースの実体に乖離することなく,常に同期された状態を維持している.

\subsubsection{決済および送金基盤 (Stripe Connect)}
CtoCプラットフォームにおける金銭取引の要となる決済インフラには,Stripe Connect \cite{stripe} を採用した.本システムでは単なる決済代行だけでなく,売り手(クリエイター)への送金管理,本人確認,および複雑なオークション決済を実現するために,以下の高度な機能群を実装している.

\paragraph{連結アカウントによるユーザー管理}
プラットフォームに参加するクリエイター(売り手)は,Stripeの「連結アカウント」として管理される.User モデルの stripeAccountId カラムには,Stripe側で発行された一意のアカウントID(acct\_...)が保存される.
これにより,売上の入金先銀行口座の管理や,特定商取引法に基づく本人確認手続きといったセンシティブな業務をStripeの堅牢なインフラにオフロードしつつ,プラットフォームとしてのコンプライアンス遵守を実現している.

\paragraph{支払いと送金の分離}
資金フローにおいては,買い手からの支払いを即時に売り手へ送金するのではなく,「支払いと送金別方式」を採用した.
この方式では,買い手の決済は一度プラットフォームのアカウントに対して行われ,その後,任意のタイミングで売り手の連結アカウントへ送金が実行される.これにより,以下の要件を満たしている.
\begin{description}
	\item[エスクロー機能] 商品の発送完了や受け取り評価が行われるまで,プラットフォーム上で売上を一時的に保持し,トラブル時の返金を容易にする.
	\item[収益分配の柔軟性] プロジェクト主催者への手数料やプラットフォーム手数料を差し引いた上で,正確な金額をクリエイターに分配する.
\end{description}

\paragraph{オークションにおける与信枠の活用}
イングリッシュオークションや抽選販売において,最大の課題である「当選後の未払い」を防ぐため,クレジットカードの「オーソリゼーション」機能を活用した決済フローを構築した.
\begin{description}
	\item[与信枠確保]  入札や抽選応募の時点で,PaymentIntent を作成し,capture\_method: 'manual' オプションを用いてカードの与信枠のみを確保する.この段階では実際の請求は発生しない.
	\item[確定と解放]落札や当選が確定した時点で capture を実行し,売上を確定させる.一方,落選や高値更新が発生した場合は即座に cancel を実行し,与信枠を解放する.
\end{description}
この仕組みにより,ユーザーの資金拘束を必要最小限に抑えつつ,システム側は「支払い能力のある入札」のみを受け付けることが可能となり,取引の安全性を飛躍的に高めている.

\subsection{データモデル設計}
本システムのデータモデルは,User,Item,Project の3つのコアエンティティを中心に設計されている.主要なエンティティの関係性をfigure 1に示す.

\begin{figure}[t]
	\centering
	\begin{tikzpicture}[
			node distance=1.8cm,
			entity/.style={draw, rectangle, rounded corners, align=center, minimum width=2.5cm, minimum height=1.2cm, fill=white, thick},
			attribute/.style={draw, ellipse, align=center, minimum width=2cm, minimum height=0.8cm, fill=white},
			relationship/.style={draw, diamond, aspect=2, align=center, minimum width=2cm, minimum height=1cm, fill=white, thick},
			arrow/.style={-Latex, thick},
			line/.style={thick}
		]
		% Entities
		\node[entity] (user) {\textbf{User}};
		\node[entity, right=4cm of user] (item) {\textbf{Item}};
		\node[entity, below=3cm of user] (project) {\textbf{Project}};

		% Relationships
		% User - Item (Created By)
		\draw[arrow] (user) -- node[above] {作成 (1:N)} (item);

		% User - Project (Owner)
		\draw[arrow] (user) -- node[left] {所有 (1:N)} (project);

		% Project - Item (Contains)
		\draw[arrow] (project.east) -- node[right] {包含 (1:N)} (item.south);

		% User - Project (Collaborates)
		\draw[line, dashed] (user.south east) -- node[sloped, above, font=\footnotesize] {参加 (M:N)} (project.north east);

		% Attributes (Selected)
		\node[attribute, above left=0.5cm of user] (uid) {id, publicId};
		\node[attribute, left=0.5cm of user] (ustripe) {stripeAccountId};
		\draw[line] (user) -- (uid);
		\draw[line] (user) -- (ustripe);

		\node[attribute, above right=0.5cm of item] (iid) {id, price};
		\node[attribute, right=0.5cm of item] (inarrative) {description\\(Narrative)};
		\draw[line] (item) -- (iid);
		\draw[line] (item) -- (inarrative);

		\node[attribute, below left=0.5cm of project] (pid) {id, mode};
		\draw[line] (project) -- (pid);

	\end{tikzpicture}
	\caption{User, Item, Project のER図(主要なリレーションのみ抜粋)}
	\label{fig:er_diagram}
\end{figure}

\subsubsection{Userモデル}
User モデルは,本プラットフォームにおけるすべての活動の基点となる「主体」を表すリソースである.単なる認証用のアカウント情報とは異なり,インターネット上におけるクリエイターとしてのアイデンティティと,経済活動を行うための法的人格としての側面を併せ持つ.

\paragraph{公開識別子とポートフォリオ機能}
各ユーザーには,データベース内部の主キー(id)とは別に,外部公開用の識別子として publicId(CUID)が付与される.これにより,https://domain/[userPubId] という予測不可能な固有URLが生成される.このURLは,ユーザーの出品作品や参加プロジェクト,活動履歴を集約したポートフォリオページとして機能し,クリエイターのブランディングを支援する.

\paragraph{経済的主体としての属性}
CtoC取引における売り手としての機能を果たすため,Stripe ConnectのアカウントIDを stripeAccountId カラムに保持する.これは,本人確認が完了したユーザーにのみ発行されるIDであり,このフィールドの有無によって,システムはユーザーが「出品可能」か「購入専用」かを判別する.また,買い手としての情報は stripeCustomerId に紐づけられ,クレジットカード情報等の機微な決済情報はすべてStripe側のVaultに保存される設計としている.

\paragraph{多層的なリレーション}
Userは他のリソースと多層的な関係を持つ.自身が作成した作品(items)だけでなく,他者と共同制作した作品(contributingItems)や,主催する企画(ownedProjects),参加者として関わる企画(contributingProjects)へのリレーションを保持し,個人の活動の広がりをグラフ構造として表現する.

\subsubsection{Itemモデル}
Item モデルは,取引の対象となる「作品」を管理するリソースである.既存のECサイトにおける「商品」モデルと決定的に異なる点は,物理的なスペック情報よりも「ナラティブ」の記述と保存に重きを置いている点である.

\paragraph{ナラティブの保存}
作品の背景ストーリーや制作過程を記録するために,description フィールドにはMarkdown形式のテキストデータが保存される.フロントエンド側ではこれがリッチテキストとしてレンダリングされ,画像や動画の埋め込みとともに,作品単体で完結した「記事」として表示される.これにより,物理的な展示会場においてQRコード経由でアクセスされた際,鑑賞者に深い理解を促すキャプションボードとしての役割を果たす.

\paragraph{販売ロジックの抽象化}
一点物や限定品など,作品の性質に応じた最適な販売形態を選択可能にするため,salesMethod カラム(ENUM型)を定義している.
\begin{description}
	\item[AUCTION] イングリッシュオークション形式.currentPrice や endTime と連動し,入札を受け付ける.
	\item[LOTTERY] 抽選販売形式.事前与信を用いた応募ロジックが適用される.
	\item[FIXED\_PRICE] 定額販売形式.即時購入が可能となる.
\end{description}

\paragraph{ステータス管理}
作品のライフサイクルを管理するために,status カラム(ENUM型)を用いる.出品前の DRAFT,販売中の FOR\_SALE,取引成立後の SOLD などの状態遷移を厳密に定義することで,不正な購入や二重決済を防ぐ排他制御を実現している.また,Stripeの商品マスタと同期するために stripeProductId および stripePriceId を保持し,決済システムとの整合性を保っている.

\subsubsection{Projectモデル}
Project モデルは,複数の User と Item を束ねる「共同体」および「文脈」を提供するリソースである.これは単なるカテゴリ分類タグではなく,独自の経済圏を持つ「企画」そのものをデジタル上に実体化させたものである.

\paragraph{ガバナンス構造の表現}
\begin{description}
	\item[Owner] プロジェクトの作成者であり,管理者権限を持つ.ownerId 外部キーによってUserモデルと一対多で結ばれる.
	\item[Collaborators] 企画に参加するクリエイター群.中間テーブルを介した多対多のリレーションによって定義される.
	      この構造により,招待制のサークル活動や,主催者が権限を持つコンテストなど,多様な組織形態をシステム上で再現可能にしている.
\end{description}

\paragraph{経済圏としての機能}
Projectは,紐づけられた作品群(items)の集合体として機能する.Item モデルは projectId を外部キーとして持ち,自身がどの文脈に属しているかを示す.このリレーションに基づき,システムは「企画単位での売上集計」や「主催者への手数料分配」といった高度な金銭処理を実行する.すなわち,Projectモデルは単なるグルーピング機能ではなく,収益分配の計算単位としても機能する設計となっている.

\subsection{Project管理方式}
本プラットフォームにおける Project は,単なる商品カテゴリや検索タグではなく,独自の規律と経済ルールを持つ「共同体」として定義される.
現実世界の創作活動には,大学のサークル活動のような水平的な人間関係に基づくものから,コンテストのように主催者が強い権限を持つ垂直的な構造,あるいはインターネット上のミームのように不特定多数が自律的に参加するものまで,多種多様な形態が存在する.
これらを単一のシステム仕様でカバーすることは困難であるため,本システムでは Project リソースに対して3つの異なるガバナンスモード(管理方式)を実装した.

\subsubsection{Cooperative Mode}
\paragraph{意義と適用領域}
Cooperative Mode は,既知のメンバー間の「信頼」を基盤とした小規模から中規模の共同体を再現するためのモードである.大学のサークル活動,グループ展,あるいは特定のクリエイター同士のコラボレーション企画などを想定している.ここでは,厳格な管理コストを削減し,メンバー間の自律的な調整を優先する運用がなされる.


\paragraph{仕様詳細}
\begin{description}
	\item[参加権限] 「招待制」を採用する.プロジェクトオーナーまたは既存メンバーからの招待リンク(署名付きURL)を受け取ったユーザーのみが,Collaboratorとして参加できる.
	\item[作品管理] 参加メンバーは,自身の判断で作品(Item)をProjectに紐づけ,即座に公開することができる.オーナーによる事前承認プロセスは存在しない.
	\item[収益構造] 原則として,売上は各作品の制作者に直接帰属する.オーナーによる「場所代」の強制徴収機能は無効化されており,金銭的な搾取が発生しない構造となる。
\end{description}

\subsubsection{Managed Mode}
\paragraph{意義と適用領域}
Managed Mode は,明確な主催者が存在し,品質管理やレギュレーションの遵守が求められる階層型の共同体である.企業主催のコンテスト,有名インフルエンサーによる企画展,アンソロジーの編纂などを想定している.
このモードの核心は,主催者に対して「管理権限」と引き換えに「経済的インセンティブ」を与える点にある.これにより,従来はボランティアに依存していた企画運営業務(キュレーション,広報,進行管理)を,収益を生む事業として成立させることを可能にする.

\paragraph{仕様詳細}
\begin{description}
	\item[参加権限] 「公募制」を採用する.ユーザーはProjectに対して参加申請を行い,オーナーが管理画面でこれを「承認」または「拒否」することでメンバーシップが確定する.
	\item[作品管理] 出品された作品は一時的に PENDING ステータスとなり,オーナーが内容を確認し承認するまで公開されない.また,オーナーは規約違反の作品を強制的にプロジェクトから除外する権限を持つ.
	\item[収益構造] Stripe Connectの "Separate Charges and Transfers" APIを活用し,「主催者手数料」 の設定を可能にしている.取引成立時,システムは売上総額からプラットフォーム手数料と主催者手数料を自動的に控除し,残額を作品制作者へ送金する.
\end{description}

\subsubsection{Theme Mode}
\paragraph{意義と適用領域}
Theme Mode は,中央集権的な管理者を排し,共通の文脈やテーマの下に不特定多数の主体が自律的に集合する「創発的な共同体」である.インターネット上で突発的に発生するトレンドや,特定のハッシュタグを介して形成される大規模な社会現象,あるいは季節的なイベントに伴う集合知的な制作活動をモデル化している.
本モードの設計目的は,参加における許認可プロセスを完全に撤廃することで参入障壁を最小化し,ネットワーク外部性を最大限に作用させることにある.これにより,短期間で幾何級数的な作品数の増加とトラフィックの凝集を実現する.

\paragraph{仕様詳細}
\begin{description}
	\item[参加権限] 「自由参加」を採用する.アカウントを持つ全ユーザーは,許可を得ることなく即座にProjectに参加できる.
	\item[作品管理] 紐づけられた作品は即時公開される.オーナー(発起人)は,通報対応を除き,他者の作品を削除する権限を持たない.
	\item[収益構造] 主催者手数料の設定は不可である.売上はすべて作品制作者に還元される.これにより,発起人が「他人の作品で金儲けをしている」という批判を避けることができ,純粋なムーブメントとしての成長を促進する.
\end{description}

\subsubsection{Project 管理方式の比較}
各モードにおける権限と金銭的フローの差異を示す。
\begin{table}[h]
	\caption{インダストリアルアートの授業}
	\label{tab:sample}
	\centering
	\begin{tabular}{lccccc} % l:Left, c:Center, r:Right 寄せ
		\hline
		方式名         & 参加フロー       & 作品公開フロー     & 主催者権限    & 収益分配(Organizer Fee) \\
		\hline \hline
		Cooperative & 招待のみ        & 即時公開        & メンバー除名可  & 不可                  \\
		Managed     & 申請 $\to$ 承認 & 投稿 $\to$ 承認 & 作品削除・否認可 & 可                   \\
		Theme       & 自由参加        & 即時公開        & 管理不可     & 不可                  \\
		\hline
	\end{tabular}
\end{table}

\subsection{統合型検索システム}
本プラットフォームにおける検索機能は,単に特定の物品を発見するためのツールではなく,ユーザーを未知の作品やクリエイター,あるいはコミュニティへと導く「ディスカバリー」の中核機能として位置づけられる.
ユーザーの探索意図は,「特定の作品が欲しい」だけでなく,「面白い企画はないか」や「特定の作家の活動を知りたい」など多岐にわたる.これら多様な検索意図を単一のインターフェースで充足させるため,User,Item,Project の3つの主要リソースを横断的に検索する「統合型オムニサーチシステム」を実装した.

\subsubsection{設計思想: 探索的検索への対応}
従来のEコマースサイトにおける検索システムは,SKU(最小管理単位)ベースの商品検索に特化しており,ユーザーが明確な購買目的を持っていることを前提とすることが多い.
しかし,一点物や創作活動を扱う本プラットフォームにおいては,ユーザーは必ずしも具体的な商品名を検索するとは限らない.「折り紙」「ガレージキット」といったジャンル名や,「学園祭」といったイベント名から検索を開始し,そこから関連するクリエイターや作品群へと興味を広げていく「探索的検索」の行動様式が支配的である.
したがって,検索システムは,単一のリソースリストを返すのではなく,システム内の多様なエンティティ(主体,文脈,作品)を網羅的に提示し,ユーザーの回遊を促す情報アーキテクチャを持つ必要がある.

\subsubsection{アーキテクチャ: 並列クエリ実行モデル}
オムニサーチのバックエンド処理においては,異なるデータ構造を持つ複数のテーブルに対する検索を効率的に処理するため,並列クエリ実行モデルを採用した.
具体的には,ユーザーが入力した単一の検索クエリに対し,バックエンド(NestJS)は以下の処理を非同期かつ並列に実行する.
\begin{description}
	\item[Project Search] プロジェクト名および概要文に対する全文検索またはベクトル検索.
	\item[User Search] ユーザー名およびプロフィール文に対する検索.
	\item[Item Search] 作品名,説明文(Markdown),タグに対する検索.
\end{description}
これらのクエリは Promise.all 等を用いて並行してデータベース(PostgreSQL)に発行され,個別の検索結果が得られ次第,単一のレスポンスオブジェクトに統合されてクライアントへ返却される.これにより,リソースごとに個別の検索画面を遷移する必要をなくし,シームレスな検索体験を実現している.

\subsubsection{インターフェース: 階層的情報の統合表示}
検索結果の表示画面においては,情報の重要度とユーザーの認知負荷を考慮し,リソースタイプごとに優先順位を設けた階層的表示を採用した.
\begin{description}
	\item[上部: 関連プロジェクト] 検索結果の最上部には,クエリに関連性の高い Project を最大3件表示する.これは,個別の作品よりも「企画」や「イベント」という大きな文脈を優先して提示することで,ユーザーをコミュニティ全体へと誘導するためである.
	\item[中部: 関連クリエイター] 次に,関連する User を最大5件表示する.これにより,特定の作家を探しているユーザーのナビゲーション意図に即座に応える.
	\item[下部: 関連作品] 最下部には,関連する Item をグリッド形式で表示する.ここでは無限スクロールを採用し,ユーザーが満足するまで作品を閲覧し続けられるよう設計している.
\end{description}
また,検索結果画面のサイドバーには,「ファセットナビゲーション」を配置した.ユーザーは,統合検索によって得られた広範な結果から,価格帯,カテゴリ,販売方式(オークション・定額など)といった属性に基づいて動的に絞り込みを行うことができる.これにより,探索的な「広げる検索」と,条件を特定する「絞る検索」の両立を図っている.

\subsection{販売形態と価格決定アルゴリズム}
本プラットフォームが扱う「一点物」や「希少在庫」の流通において,固定価格での販売は必ずしも最適解ではない.需要が供給を大幅に上回る場合,価格メカニズムが機能せず「早い者勝ち」となることで,適正価格の販売ではなくなるからである.
この課題を解決し,作品を最も必要とするユーザーへ適正価格で配分するために,本システムでは以下の4つの販売アルゴリズムを実装した.また,全ての方式において,Stripe Connect APIを用いたクレジットカードの与信枠確保を必須とすることで,CtoC取引における最大の問題である支払い不履行を低減している。

\subsubsection{イングリッシュ・オークション}
在庫が単一($N=1$)である一点物の作品に対し,最も高い評価額を持つ購入希望者を決定するための標準的な競り上げ方式である.

\paragraph{自動入札アルゴリズム}
ユーザーの利便性とサーバー負荷の軽減を考慮し,Vickrey Auctionの概念を取り入れた自動入札システムを採用した.
ユーザーは「現在価格」ではなく,自身の「支払ってもよい最高額」を入力する.システムは,2番目に高い入札額(または開始価格)に最小入札単位を加えた額を「現在価格」として自動的に算出・更新する.これにより,ユーザーは画面に張り付いて再入札を繰り返す必要がなくなり,合理的な入札行動が促進される.

\paragraph{連続的な与信枠管理フロー}
本システムでは,「最高入札者は常に支払い能力を保証されている」という状態を維持するため,最高入札者が入れ替わるたびに決済リソースの「ハンドオーバー」を行う.具体的なトランザクションフローは以下の通りである.
\begin{description}
	\item[1. 新規与信] 新たな高値入札者が現れた際,システムはそのユーザーのクレジットカードに対し,入札額全額の与信枠確保(paymentIntents.create with capture\_method: 'manual')を試行する.これに失敗した場合,入札自体を却下する.
	\item[2. DB更新] 与信確保に成功した場合のみ,データベース上の Item レコード(現在価格,リーダーID)および Bid レコードを更新する.
	\item[3. 旧与信解放] 更新完了後,直前の最高入札者の与信枠(PaymentIntent)に対してキャンセル処理(paymentIntents.cancel)を非同期的に実行し,枠を即時解放する.
\end{description}
このプロセスにより,オークション終了時に「落札者が支払えない」というリスクを回避している。

\subsubsection{複数点式オークション}
在庫が複数($N > 1$)存在するが,需要がそれを上回る場合(例:限定5個のガレージキット)に適用される,本システム独自の販売方式である.従来の単一価格オークションではなく,入札額がそのまま支払額となる Pay-as-Bid方式 を採用した.
\paragraph{当選者決定ロジック}
入札額の高い順にソートを行い,上位 $N$ 名を当選者として確定する.
\[
	B_{1} \ge B_{2} \ge \dots \ge B_{N} \ge \dots \ge B_{M}
\]
ここで $B_{i}$ は $i$ 番目のユーザーの入札額であり,上位 $N$ 名までの入札が有効となる.同額入札が存在する場合は,先着優先として順位付けを行う.
\paragraph{Pay-as-Bidによる収益最大化}
当選者全員が $B_{N}$(ボーダーライン価格)を支払う単一価格方式ではなく,各当選者が自身の提示した $B_{i}$ を支払う Pay-as-Bid 方式を採用した.
これにより,作品に対し強い選好を持つ熱狂的なファン($B_{1}$)からは高い収益を,学生などの価格に敏感な層($B_{N}$)からは適正な収益を得ることが可能となる.結果として,同一の商品であっても購入者の支払意思額に応じた価格差別化が自然発生し,クリエイターへの還元額を最大化する.
\paragraph{取引および決済フローの詳細}
本方式では,複数の当選候補者が並存するため,イングリッシュ・オークションのような「入れ替え」ではなく「全件確保」に近い戦略をとる.
\begin{description}
	\item[Step 1: 入札と与信確保 ] ユーザーが入札を行う際,システムは即座に入札額全額の与信枠確保を実行する.この時点で決済は確定しないが,カードの利用枠は押さえられる.これにより,冷やかしや入札後の支払い拒否を未然に防ぐ.
	\item[Step 2: ボーダーラインの可視化と競争] オークション期間中,入札画面には「現在,上位 $N$ 位に入るための最低価格($B_{N}$)」がリアルタイムで表示される.ユーザーはこのボーダーラインを参考に,自身の予算内で入札額を調整する.
	\item[Step 3: 決着と一括決済] 決着と一括決済 (Settlement)終了時刻(endTime)に到達した時点で,システムは全入札を確定させる.
	      上位 $N$ 名: 確保していた与信枠に対し capture を実行し,売上を確定させる.
	      落選者 ($N+1$ 位以下): 確保していた与信枠に対し cancel (Void) を実行し,即座に解放する.
\end{description}
\subsubsection{抽選販売}
価格の高騰を抑制したい場合や,機会の平等を重視する場合に用いられる方式である.技術的な特異点は,応募プロセスにおける事前与信の導入にある.
\paragraph{フローとステート管理}
\begin{description}
	\item[応募時] ユーザーが「応募」ボタンを押下した瞬間に,商品価格分の与信枠確保を実行する.確保に失敗した場合,応募は受理されない.
	\item[抽選時] 擬似乱数生成器を用いて当選者を選出する.
	\item[確定時] 当選者の取引ステータスを CAPTURED に更新し,決済を確定させる.同時に,落選者に対しては VOID 処理を行い,与信枠を即時解放する.
\end{description}
この仕組みにより,従来のアナログな抽選販売で頻発していた「当選連絡後の音信不通」や「支払い拒否」による再抽選の手間を完全にゼロにし,運用コストを最小化している.

\subsubsection{定額販売}
在庫が無制限,あるいは需給バランスが安定している作品向けの,一般的なECと同様の販売方式である.即時決済(Capture)によって取引が成立する.

\subsection{テストデータに対する要件と課題}
このような「文脈重視型」かつ「複雑なリレーションを持つ」プラットフォームの開発において,テストデータの用意は大きな課題であった.

\begin{enumerate}
	\item \textbf{リレーションの整合性}:
	      ランダムなデータを生成すると,「ユーザーが未参加のプロジェクトに関連する作品を出品している」といった外部キー制約違反や論理矛盾が頻発する.これを防ぐためには,データベースの依存関係を深く理解した生成ロジックが必要となる.

	\item \textbf{コンテンツのリアリティ}:
	      UIの検証(例えば,長い説明文がレイアウト崩れを起こさないか,検索機能が適切にキーワードを拾うか)を行うためには,「Lorem Ipsum」に代表される無意味なダミーテキストではなく,ドメイン固有の用語や適切な長さを持つ「リアルなテキスト」が必要である.特に本プラットフォームでは「制作ログ」や「物語」が主要なコンテンツであるため,その質がUX評価に直結する.

	\item \textbf{多様性の確保}:
	      検索アルゴリズムやレコメンデーション機能の性能を評価するためには,データの分布に偏りがあってはならない.特定のカテゴリ(例:アクセサリー)だけにデータが集中すると,システムの性能を正しく評価できない.
\end{enumerate}

第3章で提案したデータ生成エージェントは,これらの課題,特に「整合性」と「多様性」の両立という課題を解決するために設計されたものであり,本プラットフォームはその有効性を検証するための最適なテストベッドであると言える.
