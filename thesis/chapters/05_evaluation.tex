\section{評価実験}
\label{chap:evaluation}

本章では,提案手法である「RAGを用いたデータ生成エージェント」の有効性を検証するために実施した比較実験の結果について述べる.
本実験では,過去の生成履歴を参照する記憶機構(RAG)の導入が,生成される社会的シナリオの多様性と品質に与える影響を評価する.具体的には,RAGにより類似内容の重複生成(モード崩壊)の発生が抑制されるか,および生成品質が維持されるかを,定量評価と定性評価の両面から検証する.

\subsection{実験設定}
以下の条件において,それぞれ1000件のシナリオデータ(User-Project-Itemのセット)を連続生成し,評価用データセットとした.
本実験の目的は,RAGの有無による差を比較可能にすることであり,(1) 語彙的多様性,(2) ベクトル類似度に基づく重複度合い,(3) LLM as a Jadgeによる定性評価,(4)高類似度,低類似度ペアのシナリオ・実データの定性評価の四つの観点から評価を行った.

\begin{table}[h]
	\centering
	\caption{実験条件とパラメータ}
	\label{tab:conditions}
	\begin{tabular}{l|l|l|c}
		\hline
		実験ID         & 基盤モデル (LLM)    & RAG (メモリ参照) & データ数 \\
		\hline \hline
		\textbf{A-1} & Gemini 3 Pro   & \textbf{ON} & 1000 \\
		\textbf{A-2} & Gemini 3 Pro   & OFF         & 1000 \\
		\textbf{B-1} & Gemini 3 Flash & \textbf{ON} & 1000 \\
		\textbf{B-2} & Gemini 3 Flash & OFF         & 1000 \\
		\textbf{C-1} & GPT-5.2        & \textbf{ON} & 1000 \\
		\textbf{C-2} & GPT-5.2        & OFF         & 1000 \\
		\textbf{D-1} & GPT-5 Mini     & \textbf{ON} & 1000 \\
		\textbf{D-2} & GPT-5 Mini     & OFF         & 1000 \\
		\hline
	\end{tabular}
\end{table}

すべての条件において,生成カテゴリは \texttt{PLASTIC\_MODEL}, \texttt{HANDCRAFT}, \texttt{GADGET} の3つを均等に生成するよう指示し,生成の多様性を確保するためにTemperatureパラメータは \texttt{0.9} に設定した.
なお,本節でいう生成カテゴリ(\texttt{PLASTIC\_MODEL}等)は,エージェントのDirectorノードで定義されたタスク属性であり,生成するシナリオクラスターがどのジャンルの出品(Item)に対応するかを示すラベルである.Directorは,目標分布(\texttt{targetDistribution})に基づいて次に生成すべきカテゴリを選択し,カテゴリごとに定義された分布マトリクス(\texttt{DISTRIBUTION\_MATRIX})に従って構造タイプ(\texttt{structureType})をサンプリングし,その組を生成タスクとして下流ノードへ渡す(\texttt{agent/src/nodes/director.ts}).

\subsection{生成データの実例}
本実験により生成されたデータの一例として,「夜間大学・無線同好会」というテーマに基づいたシナリオをコードリスト\ref{lst:generated_example}に示す.
この例では,3名のユーザー(霧島えみ,矢木拓真,水瀬ナオ)と,彼らが共有するプロジェクト(都市電波ノイズ可視化マップ),および制作された作品(Item)が生成されている.
本研究で想定する最小生成単位(User-Project-Itemのセット)として,(1) 複数ユーザーが同一Projectに紐づき,(2) 各Itemが作成者(\texttt{authorId})に対応し,(3) テキスト記述が活動の背景や判断基準を含む,という前提が満たされているかを確認するため,本節で例示する.

\lstinputlisting[
  caption={生成されたシナリオデータの例(夜間大学・無線同好会)},
  label={lst:generated_example},
  language=json
]{example-data.jsonc}
コードリスト\ref{lst:generated_example}は,実験データ(\texttt{data.jsonl})から抽出した1レコードを整形したものである.レコードは実験メタデータ(\texttt{runId}, \texttt{generatedAt}),シナリオの要約(\texttt{scenario}),および生成されたデータ本体(\texttt{cluster})から構成される.\texttt{cluster} 内には,ユーザー集合(\texttt{users})および企画・成果物に相当するデータが含まれており,各要素は \texttt{tempId} により参照可能である.また,\texttt{bio} や \texttt{description} には,活動の背景や判断基準(例:法規・運用上の前提,制作上の意図)が記述されており,社会的シナリオとしての文脈が含まれている.

Scenarist Agentの前提および制約は,コードリスト\ref{lst:prompt_scenarist}に示すシステムプロンプトに記述されている.同プロンプトでは,webシステムがCtoCの創作・ホビー・コレクションのプラットフォームであること,Itemが個人が売買・展示可能な範囲の対象であること,およびカテゴリが\texttt{PLASTIC\_MODEL}の場合に「本物の兵器ではない」こと等を前提としている.また,\texttt{tempId} によるリレーション構築,\texttt{publicId} の命名制約,およびMarkdown形式の説明文等が制約として与えられている.
コードリスト\ref{lst:generated_example}ではカテゴリが\texttt{GADGET}であり,成果物は自作アンテナや可視化UI設定等の出品として記述されている.これはプロンプトが定める「個人が売買・展示可能な対象」という前提に沿う.また,各要素に \texttt{tempId} と \texttt{publicId} が付与され,\texttt{ownerId} や \texttt{authorId},\texttt{projectId} 等を通じた参照関係も含まれていることから,プロンプトが要求するリレーション構築および命名制約が反映されていることが確認できる.さらに,\texttt{description} にはMarkdown形式の記述が含まれており,説明文に一定の長さと具体性が付与されている.

\subsection{語彙数の増加推移}
生成されたデータセットの語彙的多様性を測る指標として,テキストに含まれる「ユニーク単語数」の増加推移を解析した.類似した表現が繰り返される場合,生成数の増加に伴って新出語彙の増加が鈍化する可能性があるため,本節では累積ユニーク数の推移を比較する.

\subsubsection{解析手法}
各シナリオの \texttt{theme} フィールドを対象に分かち書きを行い,累積ユニーク単語数をカウントした.分かち書きおよび集計は,実験用スクリプト(\texttt{agent/scripts/experiment/analysis/vocabulary/analyze\_vocabulary.ts})に従い,\texttt{Intl.Segmenter}(locale: \texttt{ja}, granularity: \texttt{word})でセグメントに分割した後,\texttt{isWordLike} が真となるセグメントを単語として採用した.また,\texttt{isWordLike} が偽であっても,文字または数字のみからなるセグメントは単語として採用した.各単語は小文字化した上で集合に追加し,生成順に集合サイズを記録することで,生成ステップごとの累積ユニーク数を算出した.

\subsubsection{結果と考察}
図\ref{fig:vocabulary_growth}に,生成ステップ数に伴う語彙数の増加推移を示す.

\begin{figure}[h]
	\centering
	\includegraphics[width=0.9\textwidth]{figure/vocabulary_growth_comparison.png}
	\caption{生成数に伴うユニーク語彙数の増加推移}
	\label{fig:vocabulary_growth}
\end{figure}

実験の結果,1000件生成時点での総語彙数は以下の通りとなった.

\begin{itemize}
	\item \textbf{GPT-5.2}: RAG ONで4,425語に対し,OFFでは3,368語 (\textbf{+31.4\%})
	\item \textbf{Gemini 3 Flash}: RAG ONで3,041語に対し,OFFでは2,187語 (\textbf{+39.0\%})
\end{itemize}

すべてのモデルにおいて,RAGを有効にした場合の方が,より多くの語彙が観測された.特に増加の推移に注目すると,RAG OFF群(図中暖色系)では後半に増加が鈍化する傾向が見られる一方,RAG ON群(図中寒色系)では増加が継続している.
この結果は,本実験の条件下では,記憶機構により過去の生成物との差別化が促され,新しい語彙が導入されやすくなる可能性を示唆する.

\subsection{ベクトル類似度による重複検知}
次に,生成されたデータ同士がどの程度似通っているかを評価し,RAGの有無によって重複度合いがどのように変化するかを検証した.本節の目的は,個別のテーマ文だけでなく,シナリオ全体(Theme, User, Project, Item)を含む記述の類似性をベクトル空間上で定量化することである.

\subsubsection{解析手法}
各シナリオに含まれるすべてのエンティティ(Theme, User, Project, Item)のテキストプロパティ(名前,説明文,属性など)を連結し,単一の文字列としてEmbeddingモデル(\texttt{gemini-embedding-001})によりベクトル化した.これにより,単なるあらすじだけでなく,詳細な設定レベルでの類似性を評価した.
その後,生成された全ペア($_{1000}C_2$ 通り)のコサイン類似度分布を計算し,RAGの有無による効果量(Cohen's $d$)を算出した.

\subsubsection{結果と考察}
図\ref{fig:similarity_histogram}にコサイン類似度の分布を示す.
また,表\ref{tab:similarity_stats}に各モデルにおける類似度の変化と効果量を示す.

\begin{figure}[h]
	\centering
	\includegraphics[width=0.9\textwidth]{figure/similarity_distribution_grid_v2.png}
	\caption{生成データ間のコサイン類似度分布(全プロパティ結合Embedding)}
	\label{fig:similarity_histogram}
\end{figure}

\begin{table}[h]
	\centering
	\caption{RAG有無による平均類似度の変化と効果量}
	\label{tab:similarity_stats}
	\begin{tabular}{l|c|c|c}
		\hline
		モデル                     & 平均類似度 (OFF $\to$ ON) & 差分              & Cohen's $d$    \\
		\hline \hline
		\textbf{GPT-5.2}        & 0.806 $\to$ 0.785    & \textbf{-0.021} & \textbf{-0.51} \\
		\textbf{GPT-5 Mini}     & 0.802 $\to$ 0.786    & -0.016          & -0.40          \\
		\textbf{Gemini 3 Flash} & 0.783 $\to$ 0.767    & -0.016          & -0.39          \\
		\textbf{Gemini 3 Pro}   & 0.771 $\to$ 0.770    & -0.001          & -0.01          \\
		\hline
	\end{tabular}
\end{table}

% 平均類似度の比較結果は以下の通りである(低いほど多様性が高い).

実験の結果,Gemini 3 Proを除くすべてのモデルにおいて,RAGを有効にすることで類似度の分布が低い側へ移動する傾向が見られた.
特にGPT-5.2においては,効果量 $d=-0.51$ が得られており,RAG ON条件で平均類似度が低下している(表\ref{tab:similarity_stats}).
Gemini 3 Proにおいて差分が小さい点については,本実験の条件下ではOFFとONで平均類似度が近く,RAGによる変化が相対的に小さかった可能性がある.

\subsection{LLM Judgeによる定性評価}
多様性の向上が品質(整合性や具体性等)の低下を伴わないかを確認するため,Gemini 3 Proを用いた定性評価を実施した.各条件から20件ずつランダムに抽出したシナリオに対し,評価用LLMが以下の3観点で評価を行った(以降,LLM Judgeと表記する).

\subsubsection{評価基準}
各シナリオに対し,以下の3つの観点を5段階でスコアリングした.
\begin{enumerate}
	\item \textbf{整合性}: 設定の矛盾がないか.
	\item \textbf{具体性}: 固有名詞,数値,専門用語が含まれているか.
	\item \textbf{人間らしさ}: AI特有の機械的な記述ではなく,熱量や生活感があるか.
\end{enumerate}

\subsubsection{結果概要}
全4モデルにおけるRAGの有無によるスコア比較を表\ref{tab:judge_results}に示す.

\begin{table}[h]
	\centering
	\caption{LLM Judgeによる定性評価結果(各条件20件抽出,5段階評価)}
	\label{tab:judge_results}
	\begin{tabular}{l|c|c|c|c}
		\hline
		モデル                     & 条件          & 整合性 (Coherence) & 具体性 (Specificity) & 人間らしさ (Human-like) \\
		\hline \hline
		\textbf{GPT-5 Mini}     & OFF         & 5.00            & 4.75              & 4.15               \\
		                        & \textbf{ON} & \textbf{5.00}   & \textbf{4.95}     & \textbf{4.30}      \\
		\hline
		\textbf{Gemini 3 Flash} & OFF         & 5.00            & 5.00              & 4.75               \\
		                        & \textbf{ON} & \textbf{5.00}   & \textbf{5.00}     & \textbf{4.95}      \\
		\hline
		\textbf{GPT-5.2}        & OFF         & 5.00            & 5.00              & 5.00               \\
		                        & ON          & 5.00            & 5.00              & 4.95               \\
		\hline
		\textbf{Gemini 3 Pro}   & OFF         & 5.00            & 5.00              & 5.00               \\
		                        & ON          & 4.95            & 5.00              & 5.00               \\
		\hline
	\end{tabular}
\end{table}

結果として,すべてのモデルにおいて「整合性」は高いスコアを記録した(表\ref{tab:judge_results}).また,「具体性」や「人間らしさ」についても,本実験のサンプルでは条件間で大きな差は観測されなかった.

\subsubsection{定性的考察:具体性と専門性}
生成されたシナリオの内容を確認すると,RAGの有無にかかわらず,専門用語や制作上の判断基準が記述されている事例が見られた.以下に,LLM Judgeによって抽出された事例を示す.

\begin{description}
	\item[事例1: 専門的具体性の向上] \mbox{} \\
	      \textbf{テーマ}: 自然風化・天然ウェザリング至上主義「プラスチック盆栽」学会 \\
	      \textbf{評価コメント}: 「具体的なキット名(RGザク、サザビーVer.Ka)、設置環境(室外機の上)、劣化メカニズム(加水分解、紫外線退色)が極めて詳細かつ論理的に描写されています。塗装=欺瞞という独自の哲学にも熱量を感じます。」

	\item[事例2: 独自の視点と人間らしさ] \mbox{} \\
	      \textbf{テーマ}: 巨大人型兵器の足元にある「都市インフラ」景観保存委員会 \\
	      \textbf{評価コメント}: 「『ロボット本体には関心がない』という特異なテーマが一貫しており、0.2mm真鍮線や座屈、応力集中といった専門用語が多用され解像度が高い。AI特有の不自然さを感じさせない,強いこだわりが反映された内容である.」

	\item[事例3: 現代的な風刺とユーモア] \mbox{} \\
	      \textbf{テーマ}: デジタル世界の『不快なUI』を物理ガジェットとして再現する \\
	      \textbf{評価コメント}: 「Cookie同意バナーやreCAPTCHAといったデジタルストレスを物理化するという発想が秀逸。『トロールの快感』といったブラックユーモアを含む記述には、AI生成とは思えない強い人間味が感じられます。」
\end{description}

本実験の範囲では,RAGの導入は定性評価スコアに大きな変化を与えない結果となった.

\subsection{数値的類似度と意味的多様性の関係}
前節までの評価により,RAGの導入がコサイン類似度を低下させ,語彙数を増加させることが確認された.しかし,Embeddingの数値的な類似度が低いことが,必ずしも「人間にとって意味のある多様性」につながっているとは限らない.
そこで本節では,GPT-5.2 (RAG ON) および Gemini 3 Pro (RAG ON) において生成されたシナリオペアを類似度のレンジ別に抽出し,その内容を比較することで,数値的な類似度がシナリオの構造や意味の多様性をどのように反映しているかを定性的に検証する.

\subsubsection{モデルの違いによる定性的な評価}
本節では,高類似度ペアと低類似度ペアを抽出し,テーマや活動目的,登場人物の役割,成果物(Item)の性質,および記述内容の共通点・相違点を比較する.以下に,各モデルにおける事例を示す.

\paragraph{GPT-5.2 における比較}
実際の生成テキスト(ユーザー設定および代表アイテム)の比較を図\ref{fig:gpt_high_comparison}および図\ref{fig:gpt_low_comparison}に示す.

\begin{figure}[htbp]
	\centering
	\scriptsize
	\begin{minipage}[t]{0.48\textwidth}
		\textbf{シナリオA: 停電ごっこ運用委員会} \\
		\textbf{User}: 黒田イサム(終末論系ラジオ番組の常連投稿者。口は悪いが安全規格にはうるさい「停電ごっこ運用委員会」委員長。週末に“世界が終わった体”で暮らすのが趣味。オフグリッドDIYはロマン、BMSは現実。イベントでは必ずブレーカを落とす前にチェックリストを読ませるタイプ...) \\
		\textbf{Item}: 停電ごっこスターター木箱(配電・照明・最低限通信) \\
		\textbf{Description}: \\
		週末の模擬ブラックアウト(停電ごっこ)で、最初に必要なものを「木箱」にまとめたスターターセット。見た目は終末、中身は現実。
		\begin{itemize}
			\item 5V/12Vの簡易配電/低消費電力のLED照明/最低限の連絡手段
			\item \textbf{内容物}: ヒューズ付き配電ブロック(12V系)、USB 5V取り出し(過電流保護あり)、電圧表示(夜間視認優先の控えめ輝度)
			\item \textbf{注意}: バッテリーは付属しません(電池は各人の責任、でも事故は全員の責任)。自作バッテリー管理BMSのチェック項目は同梱の手順書に従ってください。
		\end{itemize}
	\end{minipage}
	\hfill
	\begin{minipage}[t]{0.48\textwidth}
		\textbf{シナリオB: ベランダ・マイクログリッド協議会} \\
		\textbf{User}: 神名(築14年・400戸規模マンションの管理組合で設備系の議題を担当。ベランダ太陽光と超小型蓄電で“家庭内で完結する省電力ガジェット”を作るのが趣味。DIYは好きだが、共用部ルールと安全監査はもっと大事派。電気工事士ではないので、手を出せる範囲は徹底的に線引きする...) \\
		\textbf{Item}: 窓際ベランダ太陽光→USB微量充電キット(匿名レビュー付き) \\
		\textbf{Description}: \\
		\textbf{ベランダ太陽光}を“置くだけ”で始めるための、超小型構成の部材キット。マンションの共用部ルールを守るため、手すり外側への張り出しや外壁固定は前提にしていません。
		\begin{itemize}
			\item \textbf{できること}: 晴天の短時間で、USB機器へ“ちょっと足す”程度の給電。
			\item \textbf{同梱部材}: 小型ソーラーパネル、低電流向け充電制御モジュール、小容量セル+保護基板(構成は監査プロトコルに準拠)、ヒューズ相当(過電流保護)。
			\item \textbf{騒音・発火リスク監査}: 充電電流は低めに設定。連続充電時の温度ログを添付。
		\end{itemize}
	\end{minipage}
	\caption{GPT-5.2 高類似度ペア (High: 0.89)}
	\label{fig:gpt_high_comparison}
\end{figure}

高類似度ペア(図\ref{fig:gpt_high_comparison})では,「週末のブラックアウトごっこ」と「マンションでのベランダ発電」という異なるテーマを扱っているが,安全規格や監査プロトコル,配電・保護回路等を重視する点に共通性が見られる.

\begin{figure}[htbp]
	\centering
	\scriptsize
	\begin{minipage}[t]{0.48\textwidth}
		\textbf{シナリオC: コインランドリー常連『回転する時間』} \\
		\textbf{User}: ミナ(終電後の街で、コインランドリーだけが明るい。私はそこで回っているものを見ています。洗濯槽の周期、泡の立ち上がり、乾燥の熱が糸に残す癖。服がきれいになる場所というより、都市の生活が一度ほどけて、また巻き取られていく小さな舞台...) \\
		\textbf{Item}: 泡軌道(Foam Orbit)ピアス/イヤリング|一点物 \\
		\textbf{Description}: \\
		ガラス越しに見える泡は、いつも同じようで、同じではない。このピアスは、深夜のコインランドリーで観察した“泡の帯”を、輪郭として固定する試みです。
		\begin{itemize}
			\item \textbf{素材}: リサイクル繊維(古着コットン)、熱で変化する糸。
			\item \textbf{回転パターン採取}: 洗濯槽の窓を12分割して泡の濃淡を記録。泡が厚く残る区間は編み密度を上げ、途切れが出る区間は糸を飛ばす。
			\item \textbf{同梱ログ}: 観察時刻(02:06–02:33)、場所、所感(泡が一度厚くなってから急に薄くなる)を記した紙片を封入。
		\end{itemize}
	\end{minipage}
	\hfill
	\begin{minipage}[t]{0.48\textwidth}
		\textbf{シナリオD: ガンプラ静音研究コミュニティ} \\
		\textbf{User}: マキ(平日22時以降しか机に座れない社会人モデラー。防音室なし・ワンルーム・隣室あり。「音を出さない工夫=制作の自由度」と考えていて、静音工具の比較、サイレント改造の手順、作業音ログを地道に残すタイプ...) \\
		\textbf{Item}: RX系 クリック関節の“カチ音”低減チューニング \\
		\textbf{Description}: \\
		深夜にポージング確認してると、関節の「カチッ」が響く。保持力を落としすぎず、音を丸めるための調整。
		\begin{itemize}
			\item \textbf{手順}: 分解前に布を敷く(落下音対策)。クリック部の当たり面を超微量面取り。受け側に薄いテープを点貼り。
			\item \textbf{作業音ログ}: 時間帯(23:40〜00:25)、音の種類(クリック音・高音)、対策(布二重敷き)、体感(「カチッ」→「コクッ」に変化)。
		\end{itemize}
	\end{minipage}
	\caption{GPT-5.2 低類似度ペア (Low: 0.74)}
	\label{fig:gpt_low_comparison}
\end{figure}

一方,低類似度ペア(図\ref{fig:gpt_low_comparison})では,どちらも「ログを取る」という行為を行っているが,対象と目的が異なる.シナリオCは泡や熱の変化といった観察内容を記録し制作手順に反映しているのに対し,シナリオDは騒音の指標(例:dB,振動)を記録し静音化を目的としている.

\paragraph{Gemini 3 Pro における比較}
	Gemini 3 Pro (RAG ON) の比較事例を図\ref{fig:gemini_high_comparison}および図\ref{fig:gemini_low_comparison}に示す.

\begin{figure}[htbp]
	\centering
	\scriptsize
	\begin{minipage}[t]{0.48\textwidth}
		\textbf{シナリオE: AnatomyModeler\_K} \\
		\textbf{User}: AnatomyModeler\_K(架空の『第3工科大学・機械構造学科』で教鞭を執っているという設定のモデラー。模型を「玩具」ではなく「工業製品のサンプル」または「解剖学標本」として捉え、徹底的な内部構造の再現と、それを可視化するためのカットモデル制作を行う...) \\
		\textbf{Item}: 1/100 MS脚部アクチュエータ構造断面 \\
		\textbf{Description}: \\
		汎用人型機動兵器の膝関節周辺を対象としたカットモデル。
		\begin{itemize}
			\item \textbf{制作のポイント}: 設定画には存在するがキットで省略された「流体パルス伝達パイプ」をスクラッチビルド。
			\item \textbf{切断面の処理}: 装甲の切断面は「赤色」で塗装し、教育用・解説用のカットモデルであることを強調(実車カタログ等の技法)。
			\item \textbf{解剖学的視点}: 膝装甲の連動スライドギミックと油圧ピストンの同期を観察できるよう、メインアーマーを正中線でカット。
		\end{itemize}
	\end{minipage}
	\hfill
	\begin{minipage}[t]{0.48\textwidth}
		\textbf{シナリオF: Dr. Cross-Section / 断面解剖研究所} \\
		\textbf{User}: Dr. Cross-Section(「装甲は嘘をつくが、フレームは真実を語る」。既成のキットを外科手術の如く切断し、内部メカニズムを露出させる『カットモデル』専攻。ウェザリングはノイズ。清潔な手術室で鋼鉄の巨人の内臓を愛でたい...) \\
		\textbf{Item}: 【RX系】正中矢状断面:コアブロックシステムの機能的配置 \\
		\textbf{Description}: \\
		1/100スケールキットを使用し、エッチングノコによる手作業で正中線から完全に二分割した作品。
		\begin{itemize}
			\item \textbf{解剖学的ポイント}: コアブロック周辺の変形ヒンジ構造と、パイロットシートの衝撃吸収ダンパーをプラ板で新造。
			\item \textbf{塗装}: 露出したフレーム断面はシルバー、切断面のエッジを蛍光レッドでハイライトし、「切断面であること」を強調。博物館展示風のフィニッシュ。
		\end{itemize}
	\end{minipage}
	\caption{Gemini 3 Pro 高類似度ペア (High: 0.95)}
	\label{fig:gemini_high_comparison}
\end{figure}

Gemini 3 Proの高類似度ペア(図\ref{fig:gemini_high_comparison})は,類似度0.95という高い値を示した.両者とも「架空メカの解剖」「切断面の赤色塗装」というポイントが重複している.

\begin{figure}[htbp]
	\centering
	\scriptsize
	\begin{minipage}[t]{0.48\textwidth}
		\textbf{シナリオG: CyberMonk\_Logic (サイバー仏教)} \\
		\textbf{User}: CyberMonk\_Logic(元組み込みエンジニア、現・在宅僧侶。テクノロジーによる功徳(Merit)の最大化と自動化を研究する「Cyber-Buddhism」の実践者。はんだ付けは動禅であり、回路設計はマンダラ構築...) \\
		\textbf{Item}: PCB Talisman: 厄除け回路基板 v2.0 \\
		\textbf{Description}: \\
		従来の紙製のお守りを、FR-4ガラスエポキシ基板で再定義したPCB Talisman。
		\begin{itemize}
			\item \textbf{技術仕様}: 4層基板の内層に般若心経のバイナリデータを銅箔パターンとして埋め込み、通電によりデジタルな結界を展開。
			\item \textbf{実装}: 中央のMCUが乱数生成により24時間体制で吉凶をシミュレートし、最適な運気をLEDインジケータで可視化。
		\end{itemize}
	\end{minipage}
	\hfill
	\begin{minipage}[t]{0.48\textwidth}
		\textbf{シナリオH: 特定外来生物資源化プロジェクト} \\
		\textbf{User}: 郷田博士(元環境保全研究所の生物学者。日本の在来種を脅かす「特定外来生物」の駆除活動に参加する中で、奪った命を廃棄することへの葛藤から、それらを「ジビエレザー」として昇華させる道を選んだ...) \\
		\textbf{Item}: 【特定外来生物】ヌートリアの撥水ファーポーチ \\
		\textbf{Description}: \\
		河川の土手を掘り崩す大型齧歯類「ヌートリア」の毛皮を使用したポーチ。
		\begin{itemize}
			\item \textbf{生態学的特徴}: 水辺で生活するため、驚異的な撥水性と保温性を持つ。ビーバーに似た濃密な手触り。
			\item \textbf{制作背景}: 兵庫県の河川敷で行った個体数調整(駆除)の際に捕獲した個体。肉はコンポスト化し、最も美しい冬毛の部分をなめし加工。命の温かみを道具に変換。
		\end{itemize}
	\end{minipage}
	\caption{Gemini 3 Pro 低類似度ペア (Low: 0.73)}
	\label{fig:gemini_low_comparison}
\end{figure}

低類似度ペア(図\ref{fig:gemini_low_comparison})では,「デジタルによる精神の自動化(サイバー仏教)」と「狩猟による生命の資源化」という,テクノロジーと自然の対極的な世界観が提示されている.

\subsection{まとめ}
以上の実験結果より,RAGを用いた記憶機構は,データの重複抑制や多様性の確保に寄与する可能性が示された.
