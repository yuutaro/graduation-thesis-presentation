\section{結論}
\label{chap:conclusion}

本研究では,一点物プラットフォームの開発において,社会的シナリオやそれを反映したスキーマを持つテストデータ生成の課題に対し,RAGを搭載したLLMエージェントシステムを提案した.また,その有効性を検証するための実験環境として,Webプラットフォームを構築した.

% 従来の統計的手法やランダム生成では再現が困難であった「社会的シナリオ」に基づく整合性の高いリレーショナルデータを,LangGraphを用いたステートマシンによって自律的に生成する手法を確立した.
従来の手法では再現が困難であった「社会的シナリオ」に基づくリレーショナルデータについて,記憶機構を組み込んだ生成エージェントシステムにより,整合性を保ったデータを自動生成する枠組みを示した.

% 提案システムの有効性を検証するために実施した比較実験(各条件1000件生成)により,以下の知見が得られた.
提案システムの有効性を検証するために実施した比較実験により,以下の知見が得られた.

\begin{enumerate}
	\item \textbf{多様性と重複抑制}: 履歴をデータベースに登録し,それを参照しながらデータを生成することで,生成されるテキストの1000件時点でのユニーク語彙数がGemini 3 Flashにおいて39.0\%増加した.また,生成された社会シナリオおよび各エンティティのテキストをエンベディングし,全ペアのコサイン類似度分布を分析した結果,GPT-5.2において効果量 $d=-0.51$ が得られた.これらの結果は,本実験条件下では,記憶機構が生成内容の重複抑制と多様性に影響しうることを示唆する.
	\item \textbf{文脈的一貫性の確認}:
	      定性評価の結果,「夜間大学の無線同好会」の事例に見られるように,Userの経歴からItemの技術的仕様に至るまで,文脈に沿った記述が含まれるデータが生成されることが確認された.
\end{enumerate}

今後の課題として,本エージェントの実践的有用性の検証が挙げられる.
本研究では,文脈を持つテストデータの生成までを主眼としたが,本来の目的はこのデータを用いてアプリケーションの高度な機能を開発することにある.
特に,文脈を考慮した検索システム,複雑なリレーションを可視化する検索UI,あるいはユーザーの行動文脈に基づいたレコメンデーションシステムといった機能は,意味のあるデータが存在しなければ開発も評価も難しい.
今後は,本エージェントによって生成されたシナリオデータを活用し,これらの機能を実際に開発・評価することで,本手法がWebサービス開発プロセスにおいてどのように寄与しうるかを検討する.

さらに,本手法の一般化も重要な課題である.
現在は,一点物売買プラットフォームという特定のドメインに特化したデータモデルを用いているが,エージェントのアーキテクチャ自体は汎用的なものである.
今後は,この枠組みを他のWebサービスや異なるドメインのアプリケーションにも適用できるよう,前提条件やデータモデル依存を整理し,適用範囲の検討を行う.
