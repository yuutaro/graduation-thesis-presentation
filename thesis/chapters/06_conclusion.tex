\section{結論}
\label{chap:conclusion}

\subsection{まとめ}
本研究では,一点物プラットフォームの開発において,社会的シナリオやそれを反映したスキーマを持ったテストデータ生成の課題を解決するために,RAGを搭載したLLMエージェントシステムを提案した.また,その有効性を検証するためのテストベッドとして,実際にモダンなWebプラットフォームを構築した.

従来の統計的手法やランダム生成では再現が困難であった「社会的シナリオ」に基づく整合性の高いリレーショナルデータを,LangGraphを用いたステートマシンによって自律的に生成する手法を確立した.

% 提案システムの有効性を検証するために実施した比較実験(各条件1000件生成)により,以下の知見が得られた.
提案システムの有効性を検証するために実施した比較実験により,以下の知見が得られた.

\begin{enumerate}
	\item \textbf{多様性の確保と衝突回避}: 履歴をデータベースに登録し,それを参照しながらデータを生成することで,生成されるテキストの1000件時点でのユニーク語彙数がGemini 3 Flashにおいて39.0\%増加した。また、生成された社会シナリオおよび各エンティティのテキストをエンベディングし、全ペアのコサイン類似度分布を分析した結果、GPT-5.2において効果量 $d=-0.51$ という結果が得られた.これにより,記憶機構が生成内容の重複を抑制し,生成されるデータ全体の多様性を向上させることが示された.
	\item \textbf{探索空間の拡張による具体性の向上}:
    定性評価の結果,「夜間大学の無線同好会」の事例に見られるように,Userの経歴からItemの技術的仕様に至るまで,整合性の取れたデータが構築されていることが確認された.
\end{enumerate}

\subsection{今後の展望}
今後の課題として,まず本エージェントの実践的な有用性のさらなる検証が挙げられる.
本研究では,文脈を持つテストデータの生成までを主眼としたが,本来の目的はこのデータを用いてアプリケーションの高度な機能を開発することにある.
特に,文脈を考慮した検索システム,複雑なリレーションを可視化する検索UI,あるいはユーザーの行動文脈に基づいたレコメンデーションシステムといった機能は,意味のあるデータが存在しなければ開発も評価も難しい.
今後は,本エージェントによって生成された大量のシナリオデータを活用し,これらの機能を実際に開発・実装することで,本手法が実際のWebサービス開発プロセスにおいて具体的にどのように貢献できるかを確認していく必要がある.

さらに,本手法の一般化も重要な課題である.
現在は一点物売買プラットフォームという特定のドメインに特化したデータモデルを用いているが,エージェントのアーキテクチャ自体は汎用的なものである.
今後は,このフレームワークを他のWebサービスや,異なるドメインを持つアプリケーションにも適用できるよう一般化し,広範な開発現場で利用可能なツールとして昇華させることを目指す.
